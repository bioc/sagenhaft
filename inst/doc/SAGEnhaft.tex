%\VignetteIndexEntry{SAGEnhaft}
%\VignetteKeyword{annotation}
%\VignettePackage{SAGEnhaft}
\documentclass[12pt]{article}
\usepackage{hyperref}
\textwidth=6.2in
\textheight=8.5in
%\parskip=.3cm
\oddsidemargin=.1in
\evensidemargin=.1in
\headheight=-.3in

\newcommand{\Robject}[1]{{\texttt{#1}}}
\newcommand{\Rfunction}[1]{{\texttt{#1}}}
\newcommand{\Rpackage}[1]{{\textit{#1}}}

\usepackage{/usr/local/lib/R/share/texmf/Sweave}
\begin{document}
\author{Tim Beissbarth}

\title{Introduction to SAGEnhaft}

\maketitle

\section{Overview}

Serial Analysis of Gene Expression (SAGE) is a gene expression profiling
technique that estimates the abundance of thousands of gene transcripts (mRNAs)
from a cell or tissue sample in parallel (Velculescu, 1995). SAGE is based on
the sequencing of short sequence tags that are extracted at defined positions of
the transcript. As opposed to DNA microarray technology SAGE does not require
prior knowledge of the transcripts, and results in an estimate of the absolute
abundance of a transcript. However, due to sequencing errors a proportion of the
low abundance tags do not represent real genes altering the ability of SAGE to
estimate the number of transcripts that have been observed. Moreover, loss of
``true''-tags due to sequencing errors will result in altered numbers for the
abundance of genuine transcripts.

\section{SAGE}

Briefly, SAGE works as follows: RNAs from either cells or tissues are converted
to double stranded cDNA which is anchored to a solid phase at the 3'~end. The
double stranded cDNA is then cleaved with a restriction endonuclease at a 4~bp
recognition sequence, most commonly CATG. The 3'~ends of these cDNA fragments
are collected and are then divided into two populations and ligated to linkers
containing a type IIS restriction endonuclease recognition sequence, where the
enzyme cleaves up to 20~bp away from their recognition site. The two populations
are ligated together and amplified by PCR, resulting in two tags orientated tail
to tail with an anchoring enzyme recognition site at either end. Two types of
SAGE libraries are commonly used, generating tags of different length,
i.e. 10~base and 17~base tags respectively, depending on the enzyme used. For
protokolls see http://www.sagenet.org/sage\_protocol.htm.

\section{Base-calling and extraction of SAGE tags}

SAGE libraries are generated from between 1,000 to 5,000 sequenced clones, with
each sequence run consisting of up to 40 tags. Automated sequencers generate a
four-color chromatogram showing the results of the sequencing gel. These
chromatograms are read by the Phred or ABI software to call bases and assign an
error estimate for each base. These two base-calling programs, the open source
program Phred and the ABI KB basecaller, distributed with the ABI 3730
sequencing machines (http://www.appliedbiosystems.com), both assign a quality
score to each sequenced base (Ewing 1998). The quality score is given as $-10
\log_{10} P_e$, where $P_e$ is the probability of a base-calling error. The
resulting Phred or ABI files are read by functions implemented in this package
which subsequently extract the ditags and tags between the anchoring enzyme
sites (CATG) in the sequence, keeping the error scores with each base. Ditags
have to be within a specified length range, e.g. 20-24 bases for 10 base tags or
32-38 bases for 17 base tags. Duplicate ditags are removed to reduce possible
PCR bias, keeping the ditag with the highest average sequencing quality. Tag
sequences with a low average sequence quality ($\le 10$) are also removed. From
experimental SAGE libraries usually 20,000-100,000 tag sequences are generated.

\section{Sequence Error correction}

Sequencing errors may bias the gene expression measurements made by SAGE. They
may introduce non-existent tags at low abundance and decrease the real abundance
of other tags. These effects are increased in the longer tags generated in
LongSAGE libraries. Current sequencing technology generates quite accurate
estimates of sequencing error rates. Here we make use of the sequence
neighborhood of SAGE tags and error estimates from the base-calling software to
correct for such errors. We introduce a statistical model for the propagation of
sequencing errors in SAGE and suggest an Expectation-Maximization (EM) algorithm
to correct for them given observed sequences in a library and base-calling error
estimates.

For details see: Statistical modeling of sequencing errors in SAGE librarie,
\textbf{Beissbarth T, Hyde L, Smyth GK, Job C, Boon WM, Tan SS, Scott HS, Speed TP},
Bioinformatics; 7.2004; 20(ISMB Supplement), in press.

\section{Comparison of SAGE libraries}

SAGE tags are assessed for differential expression between two SAGE libraries by
computing Fisher's Exact test for each unique tag. If a particular tag has count
$n_A$ in library A and count $n_B$ in library B, and if the total number of
sequences counted is $t_A$ for library A and $t_B$ for library B, then Fisher's
Exact test is computed to test for independence in the $2 \times 2$ contingency
table with counts $n_A$, $n_B$, $t_A - n_A$ and $t_B - n_B$. This results in a
$p$-value for the null hypothesis of no differential expression for each
gene. Since the tests for different tags are almost independent, the method of
Benjamini and Hochberg (1995) was used to control the false discovery rate
(fdr). Fisher's Exact test has been found to be slow to compute but an exact
binomial test proved to be an excellent approximation when $t_A$ and $t_B$ are
large and large relative to $n_A$ and $n_B$, as they are for typical SAGE
libraries. This test is defined similarly to Fisher's Exact test but with
binomial probabilities replacing the hypergeometric probabilities. We used a
vectorized version of the binomial exact test to allow rapid computation for
complete libraries. By analogy with microarray analysis the relative difference
of a tag between two libraries is summarized by an $M$ value, which is
calculated as
$\log_2(n_A+0.5)+\log_2(t_B-n_B+0.5)-\log_2(n_B+0.5)-\log_2(t_A-n_A+0.5)$, and
the mean absolute expression is summarized as an $A$ value, which is calculated
as $0.5 (\log_2(n_A (t_A+t_B)/2t_A + 0.5) + \log_2(n_B(t_A+t_B)/2t_B +0.5))$. We
call changes with a fdr of less than 0.1 significant.

\section{Example}

The E15 library was generated from posterior cortex of embryonic C57/BL6 mice at
stage E15.5. The B6Hypothal library was generated from hypothalamus of 8 week
old C57/BL6 mice.

\begin{Schunk}
\begin{Sinput}
> library(sagenhaft)
\end{Sinput}
\begin{Soutput}
Loading required package: SparseM 
[1] "SparseM library loaded"
\end{Soutput}
\end{Schunk}

\begin{Schunk}
\begin{Sinput}
> file.copy(system.file("data/E15postHFI.zip",package="sagenhaft"),
+          "E15postHFI.zip")
> E15post<-extract.lib.from.zip("E15postHFI.zip", taglength=10,
+                               min.ditag.length=20, max.ditag.length=24)
\end{Sinput}
\end{Schunk}

\begin{Schunk}
\begin{Sinput}
> E15post <- read.sage.library(system.file("data/E15postHFI.sage", 
+     package = "sagenhaft"))
> E15post
\end{Sinput}
\begin{Soutput}
# libname: E15postHFI
# nseq: 26871
# ntag: 12636
# taglength: 10
# nfiles: 1166
# date: Sun Jun  6 18:27:28 2004
# base.calling.method: seq
# remove.duplicate.ditags: TRUE
# nduplicate.ditags: 231
# remove.N: FALSE
# remove.low.quality: 10
# min.ditag.length: 20
# max.ditag.length: 24
# cut.site: catg
# EM steps: 50
# likelihood (every 10 steps): 14749.8 15205.7 15220.6 15223.8 15225 15225.5
# var (every 10 steps): 428.5 605.5 610.2 611.1 611.5 611.6
# Removed ShortLinker: 248 249.11
# Removed ShortRibosomal: 982 1042.99
# Removed ShortMitochondrial: 864 881.84
Fields:
libname nseq ntag taglength tags seqs comment
contents of field 'tags':
tag count.raw a c g t avg.ditaglength avg.error.score count.adjusted prop.estimate
contents of field 'seqs':
seq seqextra q1 q2 q3 q4 q5 q6 q7 q8 q9 q10 ditaglength file
\end{Soutput}
\begin{Sinput}
> B6Hypo <- read.sage.library(system.file("data/B6HypothalHFI.sage", 
+     package = "sagenhaft"))
> libcomp <- compare.lib.pair(B6Hypo, E15post)
> plot(libcomp)
> libcomp
\end{Sinput}
\begin{Soutput}
# name: B6HypothalHFI:E15postHFI
# ntag: 26589
# taglength: 10
# lib1: B6HypothalHFI
# nseq1: 42775
# ntag1: 18730
# nfiles: 3514
# date: Sun Jun  6 19:08:13 2004
# base.calling.method: seq
# remove.duplicate.ditags: TRUE
# nduplicate.ditags: 214
# remove.N: FALSE
# remove.low.quality: 10
# min.ditag.length: 20
# max.ditag.length: 24
# cut.site: catg
# EM steps: 50
# likelihood (every 10 steps): 36792.6 37547.8 37569.4 37574.1 37575.9 37576.7
# var (every 10 steps): 773.8 1052.4 1059.4 1060.8 1061.3 1061.5
# Removed ShortLinker: 303 303
# Removed ShortRibosomal: 2795 2911.92
# Removed ShortMitochondrial: 3959 4072.38
# lib2: E15postHFI
# nseq2: 26871
# ntag2: 12636
# nfiles: 1166
# date: Sun Jun  6 18:27:28 2004
# base.calling.method: seq
# remove.duplicate.ditags: TRUE
# nduplicate.ditags: 231
# remove.N: FALSE
# remove.low.quality: 10
# min.ditag.length: 20
# max.ditag.length: 24
# cut.site: catg
# EM steps: 50
# likelihood (every 10 steps): 14749.8 15205.7 15220.6 15223.8 15225 15225.5
# var (every 10 steps): 428.5 605.5 610.2 611.1 611.5 611.6
# Removed ShortLinker: 248 249.11
# Removed ShortRibosomal: 982 1042.99
# Removed ShortMitochondrial: 864 881.84
Fields:
name ntag taglength data comment
contents of field 'data':
tag A.adjusted count1.raw count2.raw M.raw tests.raw count1.adjusted count2.adjusted M.adjusted tests.adjusted
\end{Soutput}
\end{Schunk}

\begin{Schunk}
\begin{Sinput}
> testlib <- combine.libs(B6Hypo, E15post)
> testlib <- estimate.errors.mean(testlib)
> testlib <- em.estimate.error.given(testlib)
> tagneighbors <- compute.sequence.neighbors(testlib$seqs[,"seq"], 10,
+                          testlib$seqs[,paste("q", 1:10, sep="")])
\end{Sinput}
\end{Schunk}

\end{document}
